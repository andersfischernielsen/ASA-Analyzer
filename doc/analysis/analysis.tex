\section{Analysis}

The type of analyses we are interested in are \emph{intra-functional}, that is, we do not consider function calls. To root the design of the analysis infrastructure
we considered the four basic analyses in \cite{spa} which were also covered during the second part of the course, these are \emph{liveness}, \emph{available expressions},
\emph{very busy expressions} and \emph{reaching definitions}. 

\newpar From a pure algorithmic point of view, these four analyses are striking similar: all of them operate on sets on a data structure that has a notion of successor and predecessor 
and both use set union, intersection and difference as operations; this suggest an analysis framework with parametric analyses instead of a framework with \emph{hardcoded}
ones. This approach promotes code reuse, hides the details of the solver and underlying operations, e.g. set union, and it exposes an interface expressive enough to
write \emph{power set based} analyses.

%\subsection{Control Flow Graph}

\newpar The intermediate representation (of the program) upon which the analyses relay on is the Control Flow Graph with single-statement blocks (CFG). A \emph{Control Flow Graph} with \emph{single-statement blocks} is a digraph in which the control flow between non-control flow statements is modeled. This differ from a \emph{pure} control flow graph in the block (node) definition; in a typical control flow graph a block is a maximal sequence of linear statements. (\cite{cooper}).

\newpar CFG generation based on the parsed Abstract Syntax Tree (AST) has been implemented in Python, since existing AST-to-CFG parser libraries did not support generating single-statement blocks as required by this project. The CFG generation will generate a CFG with single-statement blocks given a generated TIP/C subset AST. 

\newpar From each node in the CFG we require an interface to its successors (forward flow), predecessors (backward flow), left-hand side variable (if any), right-hand side expressions and type of statement.
An extra requirement for the CFG is the variables set, i.e. all the variables declared in the program and the expressions set.

The \emph{working unit} of an analysis is a set of monotone functions, and each monotone function take as an argument a node in the CFG. 

To guide the construction and interfacing of the Analysis Engine (AE), we wrote \texttt{blue\_print.py} that focus on the design and inner workings of the analysis engine (represented by the class \texttt{Analysis}) while abstracting away all the implementation details about the components interfacing with it. 

\newpar At the top level the AE requires a CFG and a non-empty list of monotone functions: \texttt{analysis = Analysis(cfg, monotone\_functions)}. To find the analysis' fix point to the given CFG, is just a matter of executing: \texttt{analysis.fix\_point()}. Also we designed the AE so the writing of monotone functions is as close as the mathematical formulation as possible.

\newpar To be able to find the fix point, the AE keeps a list of all user provided monotone functions (\texttt{self.monotone\_functions}) and a list \texttt{self.\_state} of size the number of nodes in the CFG. Each entry of the \texttt{self.\_state} list holds the current result of the monotone function that analysed the node. The signature of a monotone function looks like: \texttt{def join\_least\_upper\_bound (analysis, cfg\_node)} in which \texttt{analysis} is a reference to the AE and \texttt{cfg\_node} is a reference to the CFG node to be analysed. We send a reference to the AE to the monotone function so the user can have access the
\texttt{self.\_state} variable; this is useful when a function requires information from other nodes.
To find the fix point the AE iterates over all blocks in the CFG, per each block, each monotone function is applied in the same order as they were provided by the user when the AE was constructed, a block is considered to be analysed when a monotone function returns a non \texttt{None} value. Hence, a requirement is set to all monotone functions: if a monotone function does not apply to the given CFG node, then it must return \texttt{None}. This approach can be improved by caching, per CFG node, the function that returned a non \texttt{None} value.

\newpar The \texttt{Analysis} class exposes two functions to the monotone functions: \texttt{def least\_upper\_bound(self, left, right)} and \\\texttt{def greatest\_lower\_bound(self, left, right)} corresponding to the lattice functions \emph{least upper bound} and \emph{greatest lower bound}, respectively.

As an example of a monotone function implementation, we show the JOIN function w.r.t. the least upper bound: 
\begin{minted}{python}
def join_least_upper_bound(analysis, cfg_node):
   club = frozenset()
   for successor in cfg_node.successors:
      club = analysis.least_upper_bound(club, analysis.state(successor))
   return club
\end{minted}

\newpar To find the least fix-point, we iterate over the nodes of the CFG, applying the corresponding monotone function depending on the \emph{type} of the node. A \texttt{\_state} variable keeps track of the latest monotone function result per CFG node. The least fix-point is found when the \texttt{\_state} variable does not change. The \texttt{\_state} variable is updated each time a monotone function is executed, as a consequence a small improvement on the time complexity might be achieved depending on the iteration order of the CFG nodes.

\subsection{Lattice based analyses}
To be able to handle general lattices and not only power set based analyses, certain changes must be made to the current implementation of \texttt{Analysis}. First, a new constructor must be added it will take the CFG and monotone functions as parameters but also will take as an additional parameter a list of pairs \texttt{lattice} that represent all the edges in the lattice. If $(x,y) \in$ \texttt{lattice} then $x < y$. From this list is possible to express the partial order completely as a matrix and use this matrix to calculate the least and greatest bounds. These changes are described in the \hyperref[futurework]{Future Work} section. 


