\section{Introduction}

Analyzing a program can help detect errors in the program. \todo{Elaborate}

\newpar Our analysis works on a a subset of the \textit{C} language. Our analyzer supports a subset of the \textit{TIP} language described by Schwartzbach in \cite{spa} as C. In the following, we give an overview of the supported language.

\subsection{Expressions}
The basic expressions all denote integer values:
\begin{center}
    $\begin{array}{l}{E \rightarrow { intconst }} \\ {\quad \rightarrow { id }} \\ {\quad \rightarrow E\texttt{+}E\:|\:E\texttt{-}E\:|\:E \texttt{*} E\:|\:E \texttt{/} E\:|\:E\texttt{>}E\:|\:E\texttt{==}E} \\ \quad \rightarrow\texttt{(}E\texttt{)}\end{array}$    
\end{center}

\subsection{Statements}
The simple statements are similar to C:
\begin{center}
    $\begin{aligned} S & \rightarrow i d=E \\ & \rightarrow \texttt { printf } E ; \\ & \rightarrow S\:S \\ & \rightarrow \texttt{if(}E\texttt{)}\{S\} \\ & \rightarrow \texttt{if(} E \texttt{)}\{S\} \texttt { else }\{S\} \\ & \rightarrow \texttt { while (}E\texttt{)}\{S\} \\ & \rightarrow \texttt { int } i d_{1}, \ldots, i d_{n} \end{aligned}$
\end{center}

\subsection{An Example Program}
A program implemented in the supported subset of C can be seen in Fig. \ref{exampleprogram}.

\begin{figure}[H]
    \centering
    \begin{minted}{c}
        int main()
        {
            int x;
            x = 1;
            int y;
            y = 0;
            int z;
            z = 0 + 1;
            if (y == 0) {
                while(1) {
                    x = 0;
                }
            }
            else {
                x = 1;
            }
            return x + y;
        }
    \end{minted}
    \caption{An example program in \textit{TIP}-inspired subset of C.}
    \label{exampleprogram}
\end{figure}

