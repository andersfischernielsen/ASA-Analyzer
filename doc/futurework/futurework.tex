\section{Future Work}

\newpar As previously mentioned, as of writing only set-based analyses are supported by the framework. A generalized lattice structure has not been determined by us, and a user can therefore not load arbitrary lattice structures and accompanying monotone transfer functions. This restricts which analyses can be run on the input programs greatly. The current set-based analysis implementation should be generalized in future work to support arbitrary lattices and transfer functions in order to make the framework more flexible in allowing other analyses types to be supported. 

\newpar The dynamic nature of the implementation should be made more type-safe when users provide analyses. Validating whether a given input analysis matches the expected structure of the framework should be implemented in order to reduce errors on the user's end. This could be accomplished either by dynamically verifying that members are present on the input analyses or attempting to implement the framwork in a strongly typed language, though this might not prove to be possible due to the requirement of dynamic loading.

\newpar Employing a more robust parsing implementation for parsing the input language and generating the CFG would improve the stability of the analysis. The choice of the Python package ecosystem could be reconsidered, and a different language could be chosen. Implementing a parser for a different input language could be promising, again due to the instability of the current C parsing. The Python language has elegant built-in support for parsing Python itself and writing analyses for subsets of Python could provide a positive end result. Implementing an analysis engine in a language with support for generating an AST and converting this to a CFG with relative ease for a given input language would save time when implementing the analysis, unlike this project.

\newpar Extending the input language could provide more interesting analyses provided that an extended language having constructs similar to popular programming languages in use, such as JavaScript, Java and Python\cite{github:languages}. These analyses could then provide interesting feedback for developers in practical scenarios.